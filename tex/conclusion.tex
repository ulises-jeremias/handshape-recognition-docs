We have performed experiments to evaluate the mean accuracy of Prototypical Networks and Wide-DenseNet on three handshape recognition datasets with and without data augmentation techniques.
For all datasets we found models that showed a performance on par  with or better than the state of the art.  Additionally, our results indicate that Wide-DenseNet benefits from data augmentation on all datasets except for CIARP.

 All models achieve near-perfect accuracy on CIARP.  This results show that  the dataset is too simple  as a benchmark for handshape recognition, seems while it has more samples on the other datasets (6000),  the samples are too  homogeneous and it does not have enough variation to generalize results to real-world application. Prototypical Networks  provide a new state-of-the-art accuracy on the LSA16  dataset,  surpassing all other known methods. Wide-DenseNets  also improve upon the state of the art,  and come close  to prototypical networks; by analyzing  the  variation of the accuracy  with respect to the training set size,  we can observe that  the performance gap between the two datasets  decreases sharply when the sample size increases.  We  have also obtained a new state-of-the-art on the RWTH  dataset  with Wide-DenseNet,  while Prototypical Networks also improved upon all previous results; this shows that  newer convolutional architectures can work better with  less data,  but there's still room for improvements using specialized models.

In future work, we  will focus on comparing with other datasets  to better understand the relationship between models and dataset complexities for handshape recognition.  We also see the need to compare with pre-trained models,  which are another way to alleviate the lack of data in a certain domain,  as well as methods that can  take advantage of unlabeled data.  Finally,  we will investigate the possibility of  merging data sets from different sign languages to augment the sample size,   as well as identify the types of data  augmentation that lead to  an improvement in state-of-the-art models.
