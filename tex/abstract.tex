\begin{abstract}
Advances in convolutional neural networks have  made possible significant  improvements in the state-of-the-art in image classification. However, their success on a particular field rests on the possibility of obtaining labeled data to train networks.  Handshape recognition from images,  an important subtask  of both gesture and sign language recognition,  suffers from such a lack of data.  Furthermore,  hands are highly deformable objects and require higher than normal datasets to classify.

We analyze both state-of-the-art models for image classification, as well as data augmentation schemes and specific models to tackle  problems with small datasets.  In particular,  we perform experiments with Wide-DenseNet, a state-of-the-art convolutional architecture  and Prototypical Networks,  the state-of-the-art few-shot learning meta model. In both cases, we also quantify the impact of data augmentation on model accuracy.

Our results show that…

\keywords{ sign language, hand shape recognition,convolutional neural networks,densenet,  prototypical networks, small datasets}
\end{abstract}

        