
%%%%%%%%%%%%%%%%%%%%%%% file typeinst.tex %%%%%%%%%%%%%%%%%%%%%%%%%
%
% This is the LaTeX source for then instructions to authors using
% the LaTeX document class 'llncs.cls' for contributions to
% the Lecture Notes in Computer Sciences series.
% http://www.springer.com/lncs       Springer Heidelberg 2006/05/04
%
% It may be used as a template for your own input - copy it
% to a new file with a new name and use it as the basis
% for your article.
%
% NB: the document class 'llncs' has its own and detailed documentation, see
% ftp://ftp.springer.de/data/pubftp/pub/tex/latex/llncs/latex2e/llncsdoc.pdf
%
%%%%%%%%%%%%%%%%%%%%%%%%%%%%%%%%%%%%%%%%%%%%%%%%%%%%%%%%%%%%%%%%%%%


\documentclass[runningheads,a4paper]{llncs}

\usepackage{amssymb}
\setcounter{tocdepth}{3}
\usepackage{float}
\usepackage{caption}
\usepackage{graphicx}
\graphicspath{{tex/images/}}

\usepackage[]{booktabs}
\usepackage{hyperref}
\hypersetup{
    colorlinks=true,
    linkcolor=blue,
    filecolor=magenta,      
    urlcolor=blue,
}

\usepackage{url}
\urldef{\mailsa}\path|{alfred.hofmann, ursula.barth, ingrid.haas, frank.holzwarth,|
\urldef{\mailsb}\path|anna.kramer, leonie.kunz, christine.reiss, nicole.sator,|
\urldef{\mailsc}\path|erika.siebert-cole, peter.strasser, lncs}@springer.com|    
\newcommand{\keywords}[1]{\par\addvspace\baselineskip
\noindent\keywordname\enspace\ignorespaces#1}

\makeatletter
\newcommand{\printfnsymbol}[1]{%
  \textsuperscript{\@fnsymbol{#1}}%
}
\makeatother

\begin{document}

\mainmatter  % start of an individual contribution

% first the title is needed
\title{Recognizing Handshapes using Small and Unlabeled Datasets}

% a short form should be given in case it is too long for the running head
\titlerunning{Recognizing Handshapes using Small and Unlabeled Datasets}

% the name(s) of the author(s) follow(s) next
%
% NB: Chinese authors should write their first names(s) in front of
% their surnames. This ensures that the names appear correctly in
% the running heads and the author index.
%
\author{Ulises Jeremias Cornejo Fandos \inst{1}\thanks{equal contribution}%
\and Gaston Gustavo Rios \inst{1, 3}\printfnsymbol{1} \and \\ Franco Ronchetti \inst{1} \and Facundo Quiroga \inst{1} \and Waldo Hasperué \inst{1,2}}
%
\authorrunning{Cornejo \and Rios \and Ronchetti \and Quiroga \and Hasperué al.}
% (feature abused for this document to repeat the title also on left hand pages)

\def\aa{\}}
\def\ab{\{}

% the affiliations are given next; don't give your e-mail address
% unless you accept that it will be published
\institute{
$^{1}$ Facultad de Informática, Universidad Nacional de La Plata \\
$^{2}$  Investigador Asociado - Comisión de Investigaciones Científicas (CIC) \\
$^{3}$  Becario de entrenamiento - Comisión de Investigaciones Científicas (CIC)
 \\ \email{\ab fquiroga,fronchetti,whasperue\aa @lidi.info.unlp.edu.ar}
}

%holooo
% NB: a more complex sample for affiliations and the mapping to the
% corresponding authors can be found in the file "llncs.dem"
% (search for the string "\mainmatter" where a contribution starts).
% "llncs.dem" accompanies the document class "llncs.cls".
%

\maketitle


\begin{abstract}
Advances in convolutional neural networks have  made possible significant  improvements in the state-of-the-art in image classification. However, their success on a particular field rests on the possibility of obtaining labeled data to train networks.  Handshape recognition from images,  an important subtask  of both gesture and sign language recognition,  suffers from such a lack of data.  Furthermore,  hands are highly deformable objects and therefore handshape classification  models require larger datasets.

We analyze both state-of-the-art models for image classification, as well as data augmentation schemes and specific models to tackle  problems with small datasets.  In particular,  we perform experiments with Wide-DenseNet, a state-of-the-art convolutional architecture and Prototypical Networks, a state-of-the-art few-shot learning meta model. In both cases, we also quantify the impact of data augmentation on accuracy. 

Our results show that on small and simple data sets such as CIARP,  all models and variations of achieve perfect accuracy,  and therefore the utility of the data is highly doubtful, despite its having 6000  samples. On the other hand, in small but complex datasets such as LSA16 (800 samples),  specialized methods such as Prototypical Networks do have an advantage over other methods.  On RWTH, another complex and small dataset with close to 4000 samples,  a  traditional and state-of-the-art method such as Wide-DenseNet surpasses  all other models.  Also, data augmentation consistently increases accuracy for Wide-DenseNet,  but not full  Prototypical Networks.

\keywords{ sign language, hand shape recognition,convolutional neural networks,densenet,  prototypical networks, small datasets}
\end{abstract}

        

\section{Introduction}


Hand shape recognition is a crucial component of any sign language recognition system. In recent years, new advances in machine learning   using models such as  convolutional neural networks  and recurrent neural s have improved our ability to tackle complex Recognition problems such as speech recognition. However, sign language recognition has not  been able to take advantage of most of these advances, since  the availability of labeled, quality data for training models is currently very limited \cite{}.   

In this article we propose to  employ and compare new methods devoted to deal with small and unlabeled data sets in order to improve the current state-of-the-art in hand shape recognition for sign language. 

our approach consists of combining and comparing three different techniques For improving model performance in these conditions: data augmentation,  prototipical learning and semi supervised learning.
for data augmentation we employ several... We combine datasets X and Y ...
We also employ prototypical networks to
matching Networks are a recent technique developed to take advantage of and label data.  we apply matching Networks to the rwth dataset.

\section{Datasets and Models}
\subsection{Datasets}

\subsubsection{LSA16}

This dataset contains images of 16 handshapes of the Argentinian Sign Language (LSA), each performed 5 times by 10 different subjects, for a total of 800 images. The subjects wore color hand gloves and dark clothes.

\subsubsection{RWTH}

This dataset is composed of a selection of images taken from the sign language interpreter at the German public tv-station PHOENIX. There are a total of 45 different hand signs. The interpreters wore dark clothes in front of an artificial grey background. 


\subsubsection{CIARP}

This dataset contains 6000 images with size of 38x38 acquired by a single color camera. The images were manually labeled and fit 10 classes of hand gestures. 

% TODO add reference to this figure in each datasets subsection
\begin{figure}
    \centering
    % \includegraphics{}
    \caption{Sample images from the LSA16 (first row), RWTH (second row) and CIARP (third row) datasets.}
    \label{fig:datasets}
\end{figure}

\subsection{Models}
\subsection{Prototypical Networks for Small Datasets}
\label{models:protonet}

Prototypical Networks \cite{protonet} is a meta-learning model for the problem of few-shot classification, where a classifier must generalize to new classes not seen in the training set, given only a small number of examples of each new class. The ability of a algorithm to perform few-shot learning is typically measured by its performance on n-shot, k-way classification tasks. First a model is given a query sample belonging to a new, previously unseen class. Then, it’s also given a support set, S, consisting of n examples, each from k different unseen classes. Finally, the algorithm then has to determine which of the support set classes the query samples belong to.
Schemes for few shot classification tasks like Prototypical Networks can also be of use for training small datasets where all classes are known.

Prototypical Networks applies a compelling inductive bias in the form of class prototypes to achieve impressive few-shot performance. The key assumption is made is that there exists an embedding in which samples from each class cluster around a single prototypical representation which is simply the mean of the individual samples. This idea streamlines n-shot classification in the case of $n > 1$ as classification is simply performed by taking the label of the closest class prototype.

\subsection{DenseNet}

As our state of the art model we selected DenseNet because it can handle small datasets with low error rate\cite{pmlr-v80-pham18a}.

DenseNet \cite{densenet} works by concatenating the feature-maps of a convolutional block to the feature-maps of all the previous convolutional blocks and using this value as input for the next convolutional block. This way each convolutional block receives all the collective knowledge of the previous layers maintaining the global state of the network which can be accessed.

Convolutional networks construct informative features by fusion both spatial and chanel-wise information within local receptive fields at each layer. Squeeze and excitation blocks (SE block) \cite{Hu2017SqueezeandExcitationN} focus on the chanel-wise information used in the convolutional layers. SE blocks improve the quality of representations produced by the network by modeling the interdependency between channels to perform feature recalibration. SE blocks can be included in any model that uses convolutional layers to improve its performance at low computational cost. We added SE blocks to our DenseNet model to improve its performance.

\subsection{Data Augmentation}

Image data augmentation is a set of techniques that aim at artificially augmenting the amount of data that can be obtained from the images in the dataset. These techniques modify the images in the dataset  with a set of predefined operations to create new images that can be used to train a model.  In this manner, we can compensate  for the lack of  variability in a small dataset\cite{cubuk2019autoaugment}.


\section{Experiments}
For our experiments we will use the LSA and RWTH datasets. For data preprocessing we divide each image by 255 to get them to fit a range of [0,1], then we apply normalization feature-wise substracting the mean  and dividing by the standard deviation of each feature. We made experiments with data augmentation and without it. The data augmentation used includes flipping the images, changing their shape by a factor of 0.2 and rotating a maximum of 30 degrees.
\subsection{Setup}

\subsection{Results}

\begin{table}[h!]
\centering
\begin{tabular}{ p{18em} p{7em} p{8em} p{8em}}
\toprule
\emph{Method} & \emph{LSA16} &  \emph{RWTH}  &  \emph{CIARP} \\ \midrule
CIARP pAPER (https://link.springer.com/chapter/10.1007/978-3-319-75193-1\_53)* \cite{CIARP_resultados} & - & - &  \\
DeepHand* \cite{koller2016deep} & - & 85.50 \\
VGG16* \cite{Quiroga_blablabla} & \textbf{95.92} & \textbf{82.88} \\
DenseNet \cite{} & NN.NN & NN.NN \\
Prototypical Networks \cite{} & NN.NN & NN.NN \\
DenseNet ++ \cite{} & NN.NN & NN.NN \\
Prototypical Networks ++\cite{} & NN.NN & NN.NN \\
DenseNet + Unlabeled \cite{} & NN.NN & NN.NN \\
Prototypical Networks + Unlabeled \cite{} & NN.NN & NN.NN \\
\bottomrule
\end{tabular}
\caption{Accuracy of various convolutional neural network models on two datasets: LSA16 \cite{ronchetti2016a} and handshapes from RWTH-PHOENIX-Weather \cite{koller2016deep}. Results from methods annotated with * were taken from other papers. Models with "++" used data augmentation as described in section XXXX. \label{tab:results}}
\end{table}



\section{Conclusion}

\begin{thebibliography}{4}
    \bibitem{matchingnet} \href{https://arxiv.org/abs/1606.04080}{Matching Networks for One Shot Learning}. Oriol Vinyals, Charles Blundell, Timothy Lillicrap, Koray Kavukcuoglu, Daan Wierstra.
    \bibitem{protonet} \href{https://arxiv.org/abs/1703.05175}{Prototypical Networks for Few-shot Learning}. Jake Snell, Kevin Swersky, Richard S. Zemel.
    \bibitem{omniglot} \href{https://github.com/brendenlake/omniglot}{Omniglot dataset}.
    \bibitem{densenet} \href{https://arxiv.org/pdf/1608.06993.pdf}{Densely Connected Convolutional Networks}. Gao Huang et al.
    \bibitem{densenet_cifar} \href{https://arxiv.org/pdf/1802.03268.pdf}{Efficient Neural Architecture Search via Parameter Sharing}. Hieu Pham et al.
    \bibitem{senet} \href{https://arxiv.org/pdf/1709.01507.pdf}{Squeeze-and-Excitation Networks}
    \bibitem{data_augmentation} \href{https://books.google.com.ar/books?id=rERADwAAQBAJ&pg=PA64&lpg=PA64&dq=Variance+is+the+algorithm\%E2\%80\%99s+tendency+to+learn+random+things+irrespective+of+the+real+signal+by+fitting+highly+flexible+models+that+follow+the+error&source=bl&ots=raekkRHUoA&sig=ACfU3U0m3wuVS6hW9WDDsq4SGP8rSDK6Fg&hl=es-419&sa=X&ved=2ahUKEwjw8Z73q4rjAhVvD7kGHa4jDnwQ6AEwBXoECAkQAQ#v=onepage&q&f=false}{Machine Learning with R}. Abhijit Ghatak.
    \bibitem{wide_resnet}
    \href{https://arxiv.org/pdf/1605.07146.pdf}{Wide Residual Networks}. Sergey Zagoruyko, Nikos Komodakis.
\end{thebibliography}

\end{document}
