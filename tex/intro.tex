

Hand shape recognition is a crucial component of any sign language recognition system. In recent years, new advances in machine learning   using models such as  convolutional neural networks  and recurrent neural s have improved our ability to tackle complex Recognition problems such as speech recognition. However, sign language recognition has not  been able to take advantage of most of these advances, since  the availability of labeled, quality data for training models is currently very limited \cite{}.   

In this article we propose to  employ and compare new methods devoted to deal with small and unlabeled data sets in order to improve the current state-of-the-art in hand shape recognition for sign language. 

our approach consists of combining and comparing three different techniques For improving model performance in these conditions: data augmentation,  prototipical learning and semi supervised learning.
for data augmentation we employ several... We combine datasets X and Y ...
We also employ prototypical networks to
matching Networks are a recent technique developed to take advantage of and label data.  we apply matching Networks to the rwth dataset.